\section{Rapid Information Factory (RIF)}

The SERUMS project use the \emph{Rapid Information Factory} to enforce an effective and efficient data wrangling processing for the various data formats.

The Rapid Information Factory is a processing methodology that uses design concepts adopted out of the Production Economics' Manufacturing Production methodologies by adopting the three basic manufacturing concepts into data processing:

make to stock (MTS)
predictable demand

Antonio Arreola-risa and Gregory A. DeCroix in \cite{Arreola-Risa1998} explain how effective balance between make-to-order (MTO) versus make-to-stock (MTS) for producing multiple heterogeneous products can be achieved in a shared manufacturing facility. The same principal translated via transferable learning into a data processing context. 

\begin{itemize}
    \item make to order (MTO) predictable demand for a product and then built upfront.
    \item make-to-stock (MTS) allows customers to order products built to their specifications
    \item make to assemble (MTA) hybrid of MTS and MTA in that companies stock basic parts based on demand predictions, but do not assemble them until customers place their order. 
\end{itemize}

\section{Rapid Information Engine}

We used a Dask scheduler with 8 x Dual (64 core/128 threads per socket) AMD AEPYC 7552 Processors with 64 GB Ram and four Seagate IronWolf 110 Series 1.92TB SSD Drives clusters. This enabled us to have 1024 core/2048 treads 512 GB RAM and 60 TB disk space for data lake.

The software used is Ubuntu with Charmed Kubernetes and MAAS (Metal-as-a-Service). The Dask clusters are then created on top of Kubernetes to ensure a flexible processing ecosystem.

The workcells in the factory always setup as part of the programming blueprint.



