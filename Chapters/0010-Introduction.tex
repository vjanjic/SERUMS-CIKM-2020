\section{Introduction}

%TW notes:
%find possible focus/title?? "Leveraging healthcare data warehousing and data analytics in a pandemic era" 
%%for predictions of pandemics and infectious diseases spread?
%%for the surveillance of possible new waves of pandemic Covid-19?
%%SERUMS data analytics unveiling new waves of Covid-19 pandemic? for multinational preparedness plans?

{\color{red}{VJ: What problem are we solving here? We talked today about the problem of having inaccurate data about the number of infections, which then hampers the appropriate planning of where to deploy resources etc. We can use this as a motivational story in introduction, to point out that lack of proper sharing of information between health centres can literally cost lives. The question then is how we are solving this problem? How does RIF help with this? Do we have/can we get any results that show this? If we go by this route, I propose the following organisation of the paper:
    \begin{itemize}
    \item Section 1 - Introduction: a paragraph about general COVID information, a paragraph about inaccurate information about infections/deaths, probably with some nice diagram which shows the number of reported cases is not really following any 'normal' trajectory and that we have points where nubmer of infections/deaths inexplainably drops (e.g. over weekends). Here we should also explain why this is due to lack of information sharing rather than the practice used in reporting deaths (which we cannot really influence). Then a paragraph about RIF and the SERUMS infrastructure we are using and how it helps with the problems outlined above. Then a list of research contributions.
    \item Section 2 - Rapid Information Factory and SERUMS project: Here we describe what SERUMS is about and also describe RIF in some detail. This could be our background section, if RIF was described elsewhere in research publications.
    \item Section 3 - Impact of Problems with Sharing Medical Information in Time of COVID-19: Here we elaborate what the problem is. This is a more detailed description of the problem, probably with more graphs.
    \item Section 4 - Our Solution: This is where we describe how RIF is used to help with the problems identified in introduction and expanded on in Section 3.
    \item Section 5 - Experimental Results.
    \item Section 6 - Related Work.
    \item Section 7 - Conclusions and Future Work.
    \end{itemize}

    I don't really see where the model to predict the number of infections would fit into and this doesn't sound as a particular research novelty (unless we develop a better model than the existing ones, which would probably be impossible to prove). But if we go for some other angle, that could change things.
}}

The Serums Project (Securing Medical Data in Smart Patient-Centric Healthcare Systems)\footnote{\url{https://www.serums-h2020.org/}} is an European Union Horizon 2020 research project that supports security and privacy of future-generation healthcare systems, placing the patients at the centre of future healthcare provision, enhancing their personal care and maximising the quality of treatment citizens receive. 

The Coronavirus disease (COVID-19) shows how the currently lack of integration and aggregation of healthcare data across the world.

Our hyper-scaling data crawler methodology enables processing factory that can process the data in the healthcare.

\begin{itemize}
    \item Data and information acquisition and processing (e.g., data crawling, data quality, data privacy, mitigating biases, and data wrangling)
    \item Integration and aggregation (e.g., semantic processing, data provenance, data linkage, data fusion, knowledge graphs, data warehousing, privacy and security, modelling, information credibility)
\end{itemize}

The solution enables the integration of healthcare data and knowledge for the next generation to support the sustainability, transparency and fairness of the worldwide view of the combined the healthcare.
