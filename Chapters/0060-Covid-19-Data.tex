\section{Covid-19 Issues}

Bill Gates states in his research "Responding to Covid-19 — A Once-in-a-Century Pandemic?" that the virus has ability to kill healthy adults as well as elderly people with recognise health complications. The current accepted fatality risk ratio around 0.01. Simple terms 1 in 100 infected is at risk. The virus transmission rate or reproduction number \cite{medicalnews001} is between 2.0 to 3.0. The serial interval (time between one person developing the symptoms of a condition and a second person becoming infected and developing symptoms) is < 4 days with a silent transmission rate \cite{medicalnews001} is 0.1. 
The asymptomatic carriers rate (contracted virus but display no symptoms) \cite{medicalnews001} is 0.01 and 0.03.

The incubation time concludes that the median for developing symptoms is 5.1 days, and 97.5\% of those developing symptoms do so within 11.5 days.

%% https://www.healthline.com/health/r-nought-reproduction-number#rsubsubvalues
%% R0

%% Model & Simulation??
%% here some material on SIR model
%% https://triplebyte.com/blog/modeling-infectious-diseases
%%idea: find recent mathematical models, and parameters requirements - to analyse SERUMS capability on delivering analytics

Requirements from data processing solution.
%% ?


Gates \cite{Gates2018} predicted that world needs to accelerate work on treatments and vaccines for Covid-19 

Jay van Wyk \cite{Wyk2015} states global hyper-connectivity and increased system integration generates various benefits like (worldwide growth in incomes, education, innovation, and technology). All of these growth is driven by our ability to share data and processes. The downside is that the same rapid globalisation now has repercussions of local events regularly cascading over national borders and the fallout of financial meltdowns and environmental disasters immediately affects everyone. This increasingly dynamics and turbo-charged globalisation has now destabilise our societies by carrying a virus across the globe through the same optimised transport systems that was driving our global growth till weeks ago.

%% TW: disease surveillance must appear in the paper title maybe

disease surveillance, including a case database that is instantly accessible to relevant organisations, and rules requiring countries to share information.

Gates in 1999 \cite{Gates1999}  that business develop a  “digital nervous system” that feeds metrics on the state and trends in the business. Our research has shown that a similar concept using our rapid information factory to drive a self-healing data factory can process healthcare data like the data on Covid-19.

\section{The Raw Data}

The raw base data from Johns Hopkins University dated (17 April 2020) for mortality by country \cite{jhumortality2020}

The raw data from Medical News Today \cite{medicalnews001} contains 



%%TW: have we data for confirmed cases? or other interesting data sets?

\section{Retrieve Superstep}



\section{Assess Superstep}

\section{Process Superstep}

\section{Transform Superstep}

%% Sim

\section{Organise Superstep}

\section{Report Superstep}


\section{What next?}

Bill Gates states in his research "Responding to Covid-19 — A Once-in-a-Century Pandemic?" that ...

David Heymann and Nahoko Shindo states in their research "COVID-19: what is next for public health?" \cite{Heymann2020} that ...

Greenhalgh, Trisha states in their reseach "Video consultations for covid-19" \cite{Greenhalgh2020}



%%SOME NOTES

%%TW: I saw some research from World Health Organisation - WHO - analytical models based on differential equations to simulate the effects of the disease based on simplistic models like SIR framework. These models analyse rates of individuals being susceptible, infectious or recovered (immune), considering a population (N). 
%This model can be a start.. if we could tag Serumers records based on this possible labels (through some personal data anaysis - symptoms, treatment, confirmed covid-19, etc.) then retrieve the aggregated data (counters?) - we can deliver the force of infection based on (transmission rate x percentage of infected within population)
%SIR model delivers an indicator of the progression of the disease

%%AV I like this concept ... I could perform the tagging?? Do we include it???

%%TW: the included data sets allow the automatic tagging? I guess.. you mentioned events in the TPOLE such as patient exposed to virus, patient has covid-19, person in hospital bed, etc. We just need to fit information we have to parametrise an existent model in the literature, could be an interesting exercise...  
%e.g. The SIR model captures population changes in each compartment (susceptible, infectious and recovered) analysing/modeling the progression of the disease.
%% here some material on SIR model
%% https://triplebyte.com/blog/modeling-infectious-diseases

%% TW: I think the tagging process within serums patient records is another extension of this work, and the most challenging part in the future, because we need to understand the content of records to decide in which compartment the individual can be labelled... for example, person with confirmed covid are automatically tagged as infectious, people with chronic diseases or other conditions can be highly susceptible, as well as people that presented recent data entries symptoms like fever, cough (111 calls?), etc  
%possibly need to investigate a decision model to better label records, or a decision tree? 
%tagging living patients: their behaviour, or registered symptoms, or recent health complaints, etc. There are several parameters that can refine SIR models extending them to a variety of inputs that give more clear indication of the infection spread rate... 
%%this can be a discussion on conclusion