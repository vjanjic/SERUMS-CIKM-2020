{\color{red}{VJ:
Section 3 - Impact of Problems with Sharing Medical Information in Time of COVID-19: Here we elaborate what the problem is. This is a more detailed description of the problem, probably with more graphs.
}}

\section{Impact of Problems with Sharing Medical Information in Time of COVID-19}

{\color{red}{AFV:
We need some extra references and background here
}}

09:34 UTC on 18 April 2020, a total of 2,256,844 cases are confirmed
180 countries involved
200 territories
1,530,643 active cases and 154,350 deaths

66\% fewer people would have been infected if China had implemented measures as little as a week earlier

Tested samples from 3,330 people in Santa Clara county and found the virus was 50 to 85 times more common than official figures indicated

Care England, Britain’s largest representative body for care homes, told the Daily Telegraph that up to 7,500 care home residents may have died of the virus.
%%%%%%%%%%%%%%%%%%%%%%
Track:
Be aware of systems inter-dependencies and unintended consequences

Forecast short-term demand.

Managing Supply Problems and Shifting Bottlenecks

Focus on Information, Fast Decision-Making, and Learning

\cite{Fox2010}
\cite{Matthias2016}

\cite{Costa2017}

\cite{Graban2018}
%%%%%%%%%%%%%%%%%%%%%%

\begin{itemize}
    \item Lack of Integration and aggregation
    \item Lack of data provenance, data linkage
    \item Lack of information credibility
    \item Lack of data sharing between primary and secondery health providers
\end{itemize}

\section{Impacts of Healthcare}


\begin{itemize}
    \item Lack of information credibility causes the effective planning of critical resources
    \item Lack of data provenance, data linkage and misinformation causes ineffective allowcation and delays in the critical supply-chain.
    \item This resulted in response team mobilising too late in the cycle.
\end{itemize}

