{\color{red}{VJ:
Section 2 - Rapid Information Factory and SERUMS project: Here we describe what SERUMS is about and also describe RIF in some detail. This could be our background section, if RIF was described elsewhere in research publications.
}}

\section{Rapid Information Factory (RIF)}
The Rapid Information Factory is a processing methodology that enable the rapid deployment of data engineering and data science artefacts against a set of external data sources.

{\color{red}{AFV:
AFV will place a summary of the RIF here.

And will complete the base info on RIF and Data Lake
}}

The RIF consists of two primary components:

\begin{itemize}
    \item Data Lake
    The data lake is a 
    
    The data lake is segregated into six processing zones (Work-space, Raw, Structured, Curated, Consumer and Analytic) 
    The six zones performs the following:
    \begin{itemize}
        \item Work-space - Temporary and transient storage during data wrangling and pre-processing
        \item Raw - Storage of federated sources systems' data as-is
        \item Structured - Storage of classified data with meta data to support processing
        \item Curated - Business Insights using T-P-O-L-E data vault, data warehouse and data marts
        \item Consumer - Storage of final published business insights
        \item Analytic - Storage of final published bulk data for further analyse by other systems
    \end{itemize}
    
    The inner details and workings of the data lake is explained in detail by Andreas Vermeulen in 
    
    
    \item R-A-P-T-O-R engine
    The Retrieve-Assess-Process-Transform-Organise-Report (R-A-P-T-O-R) engine is a hyper-scale distributed processing engine that uses graph theory in the form of a Directed Acyclic Graph (DAG) to form \emph{blueprints} with data crawlers as nodes artefacts. The details of this discussed by Vermeulen in "" \cite{...}
    
    Christian Krause, Matthias Tichy, Holger Giese \cite{Krause2014} states "Complex queries and updates demand expressive high-level languages which can still be efficiently executed on these large-scale graphs.". we have taken the basic concept of graphs and expanded it to have active self-healing data crawlers that are managed by these large graphs as blueprints. The R-A-P-T-O-R pipeline is a six step Bulk Synchronous Parallel (BSP) model.
    
    Krause's 48 processing nodes with 8 cores has been upgraded to 48 processing nodes with 1024 cores/2048 treads 512 GB RAM using a kubernetes based Dask Cluster with RaptorQube (a weak AI engine) managing the processing for the Covid-19 solution. 
    
\end{itemize}

The research for Covid-19 uses a NetworkX based blueprint format and dask scheduled python coded artefacts to process the Covid-19 source data and extract the research insights.

\section{SERUMS}

The SERUMS project is ...

The SERUMS project enables a ...

{\color{red}{AF:
We need a SERUMS summary here!!
}}

{\color{red}{VJ: I can do this, should be relatively easy to pull out from other SERUMS papers that we have}}



\section{RIF plus SERUMS}

The SERUMS now EU but planned global data privacy and information acquisition capability delivers the data into the Raw Zone for the RIF to process.

The RIF processing capability supports the effective and efficient pre-processing via RIF data crawlers to ensure acquisition via retrieve super-step \cite{Vermeulen2018ret} , data quality via assess super-step \cite{Vermeulen2018ass}, data wrangling via process \cite{Vermeulen2018prc} and transform \cite{Vermeulen2018trf} super-steps, data privacy via organise and report super-steps.

Our research shows the combination of SERUMS with using a RIF is a viable solution to handle the 3Vs (volume, variety, veracity and velocity) of the Covid-19 requirement to achieve value through near-real-time .

\begin{itemize}
    \item volume - 
    \item variety - no data standard
    \item veracity - data has error and invalid values
    \item velocity - Healthcare (Covid-19) is a global 24 hours a day process
\end{itemize}