\section{Introduction}

{\color{red}{VJ: What problem are we solving here? We talked today about the problem of having inaccurate data about the number of infections, which then hampers the appropriate planning of where to deploy resources etc. We can use this as a motivational story in introduction, to point out that lack of proper sharing of information between health centres can literally cost lives. The question then is how we are solving this problem? How does RIF help with this? Do we have/can we get any results that show this? If we go by this route, I propose the following organisation of the paper:

Section 1 - Introduction: a paragraph about general COVID-19 information, a paragraph about inaccurate information about infections/deaths, probably with some nice diagram which shows the number of reported cases is not really following any 'normal' trajectory and that we have points where number of infections/deaths unexplained drops (e.g. over weekends). Here we should also explain why this is due to lack of information sharing rather than the practice used in reporting deaths (which we cannot really influence). Then a paragraph about RIF and the SERUMS infrastructure we are using and how it helps with the problems outlined above. Then a list of research contributions.}}
    
Covid-19 started on ...

The observed issues:

